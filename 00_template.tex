\chapter{Experiment Name}

\section*{Objectives}

\begin{enumerate}
\itemsep0em
\item List
\end{enumerate}



\section*{Introduction/Theory/Background}

\subsection*{Subsection 1 - Figures}
\begin{figure}[!htb]
    \centering
    \includegraphics[width=0.5\textwidth]{example-image-a}
    \caption{Insert caption here}
    \label{fig:<name>}
\end{figure}

\subsection*{Subsection 2 - Multiple Figures}

\begin{figure}[!htb]
\captionsetup[subfigure]{justification=centering}
\centering
        \begin{subfigure}[b]{0.5\textwidth}
        \centering
                \includegraphics[scale=0.5]{example-image-b}
                \caption{Insert caption here}
                \label{fig:<name>}
        \end{subfigure}\hfill
        \begin{subfigure}[b]{0.5\textwidth}
        \centering
                \includegraphics[scale=0.5]{example-image-c}
                \caption{Insert caption here}
                \label{fig:<name>}
        \end{subfigure}
        \par\bigskip
\end{figure}


\begin{question}
\paragraph{Question:} Place your question here.~\\

\paragraph{Question:} You can place more than one in a box.
\end{question}

\begin{imp}
Use this box for important information.
\end{imp}

\begin{tip}
Use this box for tips.
\end{tip}

\todo[inline,color=yellow]{Test inline comment using the todonotes package}

 
\todo[color=cyan]{Test comment using the todonotes package, not inline}

 

 
\section*{Experimental Setup}


\subsection*{Apparatus}

\begin{enumerate}
\itemsep0em
\item List
\end{enumerate}



\subsection*{Description}

Put in any items you think need descriptions. You can include descriptions of different software, etc.

\subsection*{Warnings}
\begin{itemize}
\itemsep0em
\item List
\end{itemize}


\section*{Procedure}
\subsection*{Part A}

\subsection*{Part B}

\subsection*{Part C}



\section*{References}
\begin{enumerate}
\itemsep0em
\item List
\end{enumerate}

