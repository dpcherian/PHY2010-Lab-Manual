\chapter*{Preface to the first edition}

The contents of this book were written to serve as laboratory handouts for the \textsl{Classical Mechanics and Electromagnetism} laboratory at Ashoka University. These handouts were written when I was the Teaching Fellow for the first two laboratory courses at Ashoka University, from 2017 to 2020. 

Most of the experiments in this laboratory are standard and can be found in other labs of the same level elsewhere. The experiment on electromagnetic damping is an exception, and was suggested to us by Dr. Rajesh Khaparde from the Homi Bhabha Centre for Science Education and Research (HBCSE), where it was designed. We have, however, introduced modifications in the equipment and handouts that attempt to make these experiments more interesting to undergraduate students. One such addition to this laboratory is the use of \textsl{Tracker Video Analysis}, an ingenious piece of free software that allows students to collect and analyse data on the positions of moving objects by taking videos of their experiments. In this book, it is used to collect data in the experiments on free fall and electromagnetic damping. However, in the past, students have also used it to analyse data from the experiments on the air track as well as Kater's pendulum. Additionally, for the experiment on equipotential curves, our indispensable lab technician Pradip Chaudhari has fabricated a range of different electrodes using which a variety of different electrostatic configurations can be probed. 

I would like to express my gratitude to Dr. Sabyasachi Bhattacharya (then C. V. Raman Professor at Ashoka and currently Director of TCG-Crest) for the ideas and years of experience in experimental physics that he brought to the laboratory. I would also like to thank Professor Bikram Phookun without whose constant support and guidance this work would not have been possible. 

And lastly, I would like to express my appreciation to the first batch of physics students at Ashoka University who took this lab in the Monsoon of 2018 and who -- despite never having the chance to use these manuals because they weren't yet completed -- contributed greatly to my understanding of all the experiments detailed in this book. I state their names here for posterity: Aditya Singh, Aishwarya Jain, Anand Waghmare, Heer Shah, Nayanika Krishnan, Rahul Menon, Rashmi Gottumukkala, Riya Banerjee, Shwetabh Singh, Sreerag K P, Sreya Dey, Srinidhi Pithani, Vidur Singh, and Yajushi Khurana.




\hfill
\begin{tabular}{@{}l@{}}
Philip Cherian\\
Teaching Fellow (2017 -- 2020)
\end{tabular}
